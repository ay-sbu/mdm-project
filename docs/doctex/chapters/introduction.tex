
\فصل{کلیات}

‫در ‬‫این‬ ‫بخش‬ ‫یک‬ ‫معرفی‬ ‫اولیه‬ ‫از‬ ‫پروژه‬ ‫انجام‬ ‫شده‬ ‫آورده‬ ‫شده‬ ‫است. ‬‬‫در‬ ‫ادامه‪،‬‬ ‫پس‬ ‫از‬ ‫بیان‬ مسئله ‫پروژه‬ ‫و‬ ‫توضیح‬ ‫کلی‬ ‫به ‫معرفی‬ نوع روش حل مسئله می پردازیم.

\قسمت{تعریف مسئله}

در دنیای امروزی که استفاده از دستگاه‌های موبایل و تکنولوژی‌های اینترنت اشیاء \پانویس{Internet of Things} به سرعت در حال گسترش است، مدیریت امنیت و دسترسی به این دستگاه‌ها از اهمیت بسیاری برخوردار است. این مسئله بیشترین تأثیر خود را بر سازمان‌ها و بخش‌های صنعتی که بسترهای پیچیده‌ای از دستگاه‌های متصل به شبکه مدیریت می‌کنند، به ویژه در حوزه‌های حساس مانند بهداشت، تولید و امنیت دارد.
این مسئله اساساً در مرحله پیاده‌سازی و اجرای سیستم‌های مدیریت دستگاه‌های موبایلی و خدمات ابری سمت سرور برای اتصال به دستگاه‌های اینترنت اشیاء رخ داده است. با ظهور فناوری‌های جدید و افزایش تعداد و تنوع دستگاه‌های متصل، نیاز به راه‌حل‌هایی که امنیت این ارتباطات را تضمین کنند، چالش‌های بیشتری به وجود آمده است.
این مسئله از سال‌های اخیر با افزایش نیاز به استفاده از دستگاه‌های موبایلی و تکنولوژی‌های اینترنت اشیاء برای بهبود عملکرد، کاهش هزینه‌ها و افزایش بهره‌وری، به شدت به نمایش در آمده است. این نیاز باعث شده است که سازمان‌ها به دنبال راه‌حل‌هایی برای مدیریت امنیت و دسترسی به دستگاه‌های متصل باشند، به ویژه در مواجهه با چالش‌هایی همچون مدیریت مرکزی دسترسی‌ها و انطباق با معماری‌های مختلف دستگاه‌ها.
راه‌حل‌های پیشنهادی کنونی اغلب با محدودیت‌هایی مانند عدم یکپارچگی، پیچیدگی در تنظیمات و مدیریت دسترسی‌ها، و عدم انعطاف و توسعه‌پذیری روبه‌رو هستند. بهبود این راه‌حل‌ها از طریق ارائه پیاده‌سازی سیاست‌های دسترسی پویا و هوشمند و همچنین پیاده‌سازی ساده با اسناد کافی امری ضروری است تا عملکرد بهتری در این حوزه و برای نیازمندی هایی که بیان می شود فراهم شود.


\قسمت{کلیات روش پیشنهادی}

با استفاده از بررسی موارد موجود که بصورت متن‌باز وجود دارند و پیدا کردن مشکلات آنها سعی می‌کنیم که ابتدا یک معماری نرم‌افزاری مناسب با قابلیت انعطاف بالا و سادگی طراحی کنیم و سپس به پیاده‌سازی و تست آن خواهیم پرداخت.


\قسمت{ساختار پروژه}

همانطور که گفته شد در ادامه ابتدا موارد موجود و پروژه‌های مشابه را بطور کامل بررسی می‌کنیم و مشکلات آنها را پیدا می‌کنیم، سپس معماری نرم‌افزار موردنیاز و یکپارچه خودمان را با توجه به نیازها و قابلیت پیاده‌سازی و سادگی طراحی می‌کنیم و در نهایت به پیاده‌سازی و تست نرم‌افزار می‌پردازیم.
