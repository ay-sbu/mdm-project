
\فصل{مفاهیم پایه و کارهای مرتبط}

\قسمت{مقدمه}

\زیرقسمت{تعاریف و مفاهیم}

مدیریت دستگاه‌های موبایلی شامل یک سری مفاهیم پایه است که در امنیت و مدیریت دستگاه‌های موبایل و تجهیزات اینترنت اشیاء بسیار حیاتی هستند. در زیر به برخی از این مفاهیم پایه اشاره می‌شود:

احراز هویت (Authentication): احراز هویت به فرآیندی اطلاق می‌شود که در آن هویت واقعی کاربر یا دستگاه تایید می‌شود. در MDM، این فرآیند برای اطمینان از اینکه دستگاه‌های متصل به شبکه سازمانی، همان دستگاه‌هایی هستند که ادعا می‌کنند و دارای حق دسترسی به منابع سازمانی هستند، اساسی است.
مدیریت دستگاه (Device Management): مدیریت دستگاه شامل این است که سازمان‌ها قادر باشند دستگاه‌های متصل به شبکه را از راه دور مدیریت و کنترل کنند. این شامل دسترسی دادن به برخی تجهیزات در دستگاه موبایلی، تغییر برخی تغییرات، ذخیره سازی لاگ‌های کاربران و نظارت بر آنها و … می‌شود.
رمزنگاری (Encryption): رمزنگاری به فرآیند تبدیل اطلاعات به صورت ناخوانا به یک فرد غیر مجاز اشاره دارد. در مدیریت دستگاه‌های موبایلی، رمزنگاری برای محافظت از داده‌های حساس و اطلاعات شخصی کاربران در دستگاه‌ها ضروری است.
نظارت (Monitoring): به فرآیند نظارت بر وضعیت و عملکرد دستگاه‌ها می‌پردازد. این شامل نظارت بر مصرف باتری، فضای ذخیره‌سازی، نصب نرم‌افزارها، استفاده از داده و سایر ویژگی‌های دستگاه می‌شود.
پاک کردن اطلاعات (Data Wipe): این قابلیت به سازمان‌ها اجازه می‌دهد که در صورت دزدیده شدن دستگاه یا گم شدن آن، به صورت از راه دور تمامی داده‌های موجود در دستگاه را از بین ببرند. این کار باعث حفظ امنیت اطلاعات حساس سازمانی می‌شود.
خط مشی‌های دسترسی (Access Policies): سیاست‌های دسترسی تعیین می‌کنند که کاربران یا دستگاه‌های متصل به شبکه چه نوع دسترسی‌ها و مجوزهایی به منابع سازمانی دارند. این سیاست‌ها بر اساس نقش‌ها، گروه‌ها و سطوح امنیتی تعیین می‌شوند.

\قسمت{
 تحلیل نقاط قوت و ضعف منابع غیرپژوهشی مشابه
}



